\documentclass[modern]{aastex63}

\usepackage{acro}
\usepackage{amsmath}
\usepackage{amssymb}

\DeclareMathOperator{\var}{var}

\newcommand{\dd}{\mathrm{d}}
\newcommand{\diff}[2]{\frac{\dd #1}{\dd #2}}

\DeclareAcronym{PSD}{
  short = {PSD},
  long = {power spectral density}
}
\DeclareAcronym{SNR}{
  short = {SNR},
  long = {signal to noise ratio}
}

\begin{document}

\title{Optimal Detection of Stochastic Signals}
\author[0000-0003-1540-8562]{Will M. Farr}
\email{wfarr@flatironinstitute.org}
\affiliation{Center for Computational Astronomy, Flatiron Institute, New York NY 10010, United States}
\affiliation{Department of Physics and Astronomy, Stony Brook University, Stony Brook NY 11794, United States}

\begin{abstract}
%
  I derive an optimal statistic for detection of a common stochastic signal in
  independent data streams.  I compute the SNR in terms of the \ac{PSD} of the
  common signal and each stream's noise, and give several useful limits.
%
\end{abstract}

\section{Formalities}

Imagine we have a stochastic stationary zero-mean Gaussian signal, $h(t)$, whose
\ac{PSD} is $P_h(f)$:
%
\begin{equation}
P_h(f) = \lim_{T \to \infty} \frac{1}{T} \left| \int_{-T}^{T} \dd t \, e^{-2\pi f t} h(t) \right|^2.
\end{equation}
%
The signal is linearly projected into two data streams\footnote{The formalism
here generalizes easily to an arbitrary number of data streams.} with
independent stochastic stationary zero-mean Gaussian noise:
%
\begin{equation}
  s_1(t) = A_1 h(t) + n_1(t)
\end{equation}
%
and
%
\begin{equation}
  s_2(t) = A_2 h(t) + n_2(t),
\end{equation}
%
where the \ac{PSD} of each stream's noise is
%
\begin{equation}
  P_i(f) = \lim_{T \to \infty} \frac{1}{T} \left| \int_{-T}^{T} \dd t \, e^{-2\pi f t} n_i(t) \right|^2,
\end{equation}
%
and the signal and both noise streams are independent:
%
\begin{equation}
  p\left( h, n_1, n_2 \right) = p\left( h \right) p\left( n_1 \right) p\left( n_2 \right).
\end{equation}

Stationarity implies that the Fourier components are independent; Gaussianity
and the \acp{PSD} imply that
%
\begin{equation}
  \tilde{h}(f) \sim N\left[0, \sqrt{T P_h(f)} \right]
\end{equation}
%
and
%
\begin{equation}
  \tilde{s}_i(f) \sim N\left[ A_i \tilde{h}(f), \sqrt{T P_i(f)} \right],
\end{equation}
%
where $N[\mu, \sigma]$ is a Gaussian distribution with mean $\mu$ and standard
deviation $\sigma$.  Integrating out the (unknown) signal $h$\footnote{See
\citet{Cornish2013}.}, we have the marginal likelihood for the data streams
%
\begin{multline}
  \mathcal{L} \left( s_1(f), s_2(f) \mid P_h(f) \right) = \int \dd h(f) \, N\left[ A_1 \tilde{h}(f), \sqrt{T P_1(f)} \right]\left( s_1(f) \right) \\ \times N\left[ A_2 \tilde{h}(f), \sqrt{T P_2(f)} \right]\left( s_2(f) \right) N\left[0, \sqrt{T P_h(f)} \right]\left( h(f) \right).
\end{multline}
A bit of algebra reveals
%
\begin{multline}
  \mathcal{L} \left( s_1, s_2 \mid P_h \right) = \\ \frac{1}{2\pi \sqrt{S_h \left( A_2^2 S_1 + A_1^2 S_2 \right) + S_1 S_2}} \exp\left[ - \frac{\left(A_2 s_1 - A_1 s_2\right)^2 S_h + s_2^2 S_1 + s_1^2 S_2}{2 \left(\left(A_2^2 S_1 + A_1^2 S_2 \right)S_h + S_1 S_2 \right)}\right],
\end{multline}
%
where we have suppressed the dependence on $f$ for the moment, and absorbed the
$T$-dependence into $S$ via
%
\begin{equation}
  S_x \equiv T P_x
\end{equation}
%
(i.e.\ $S_x$ is the variance of $\tilde{x}$).

The maximum-likelihood estimator of $P_h$ is
%
\begin{equation}
  \label{eq:ml-est}
 \hat{P}_h \equiv \frac{1}{T} \frac{\left( A_2 S_1 s_2 + A_1 S_2 s_1 \right)^2 - S_1 S_2 \left( A_2^2 S_1 + A_1^2 S_2 \right)}{\left( A_2^2 S_1 + A_1^2 S_2 \right)^2}.
\end{equation}
%
The expected value of $\hat{P}_h$ is
%
\begin{equation}
  \left\langle \hat{P}_h \right\rangle = P_h
\end{equation}
%
(i.e.\ $\hat{P}$ is an un-biased estimator of $P_h$---whatever that is worth)
and the variance is
%
\begin{equation}
  \label{eq:est-variance}
  \var \hat{P}_h = \frac{2 \left( S_h \left(A_2^2 S_1 + A_1^2 S_2 \right) + S_1 S_2 \right)^2}{T^2 \left( A_2^2 S_1 + A_1^2 S_2 \right)^2} = \frac{2 \left( P_h \left(A_2^2 P_1 + A_1^2 P_2 \right) + P_1 P_2 \right)^2}{\left( A_2^2 P_1 + A_1^2 P_2 \right)^2}
\end{equation}
%
Note that the variance in Eq.\ \eqref{eq:est-variance} is independent of $T$, so
that the \ac{SNR} \emph{in a single frequency bin} is independent of time.

Typically we will be estimating the power in some range of frequencies, not just
in a single bin.  Because the different frequencies are independent
(stationarity), the optimal estimator for the integral of $P_h(f)$ over some
range is given by
%
\begin{equation}
  \int_{f_0}^{f_1} \dd f \, \hat{P}_h(f) \simeq \sum_{f = f_0}^{f_1} \Delta f \hat{P}_h(f),
\end{equation}
%
where $\Delta f = 1/T$ is the frequency resolution implied by an observation
over a time $T$.   The mean of this estimator is the integral of $P_h$ over the
corresponding interval; the variance of this estimator is given by
%
\begin{equation}
  \sum_{f = f_0}^{f_1} \Delta f^2 \var \hat{P}_h(f),
\end{equation}
%
and therefore the \ac{SNR} of such a measurement is given by
%
\begin{equation}
  \label{eq:finite-bandwidth-snr}
  \rho = \frac{\sqrt{T} \int_{f_0}^{f_1} \dd f \, P_h\left( f \right)}{\sqrt{\int_{f_0}^{f_1} \dd f \, \var \hat{P}_h(f)}}
\end{equation}
%
with $\var \hat{P}_h$ given by Eq.\ \eqref{eq:est-variance}.  We see that the
\ac{SNR} for a measurement over some finite bandwidth grows with $\sqrt{T}$, as
it should.

\section{Useful Limits}

\subsection{Signal-Dominated Limit}

If $A^2_{1,2} P_h \gg P_{1,2}$ over the relevant range of frequencies, we have
%
\begin{equation}
  \var \hat{P}_h \simeq 2 P_h^2,
\end{equation}
%
and the per-bin \ac{SNR} asymptotes to $1/2$.  The finite-bandwidth \ac{SNR},
Eq.\ \eqref{eq:finite-bandwidth-snr}, becomes
%
\begin{equation}
  \rho \simeq \sqrt{\frac{T \left( f_1 - f_0 \right)}{2}} \frac{\left\langle P_h \right\rangle}{\sqrt{\left\langle P_h^2 \right\rangle}}
\end{equation}
%
where angle brackets indicate an average over frequencies $f_0 \leq f \leq f_1$.
Note that in this limit the \ac{SNR} is independent of the amplitude of $P_h$.

\subsection{Noise-Dominated Limit}

In the opposite limit, $A_{1,2} P_h \ll P_{1,2}$, we have
%
\begin{equation}
  \var \hat{P}_h \simeq \frac{2 P_1^2 P_2^2}{\left( A_2^2 P_1 + A_1^2 P_2\right)^2} = \frac{2}{\left( A_1^2 / P_1 + A_2^2/P_2 \right)^2},
\end{equation}
%
and the finite-bandwidth \ac{SNR} becomes
%
\begin{equation}
  \rho \simeq \sqrt{\frac{T\left(f_1 - f_0 \right)}{2}} \frac{\left\langle P_h \right\rangle}{ \sqrt{\left\langle \frac{1}{\left( A_1^2 / P_1 + A_2^2/P_2 \right)^2}\right\rangle}}
\end{equation}
%
and the \ac{SNR} is reduced compared to the signal-dominated case by a factor
%
\begin{equation}
  \label{eq:approx-snr-reduction}
  \alpha \sim \frac{A_1^2 P_h}{P_1} + \frac{A_2^2 P_h}{P_2}.
\end{equation}
%

\bibliography{StochasticStats}

\end{document}
